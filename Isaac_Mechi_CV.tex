\documentclass[a4paper]{article}

\usepackage{tabularx}
\usepackage [portuguese,brazilian]{babel}     % nomes e hifena��o em portugu�s
\usepackage{t1enc}              % Permite digitar os acentos de forma normal
\usepackage[utf8]{inputenc}

\usepackage{ae}
\usepackage[T1]{fontenc}
\usepackage{CV}


\newcommand{\til}{\~{}}
\begin{document}

\pagestyle{empty}

%Ueberschrift
\begin{center}
\huge{\textsc{Curriculum Vitae}}
\vspace{\baselineskip}

\Large{\textsc{ISAAC MECHI}}
\end{center}
\vspace{1.5\baselineskip}

\section{Endereço}

\begin{flushleft}
	Rua Amália Forti Poli, nº 483 – Jd. Do Lago II Continuação \\
	CEP 13051-059 Campinas - SP
\end{flushleft}


\section{Dados Pessoais}
\begin{flushleft}
  Idade: 26 \\
  Sexo: Masculino \\
  Data de Nascimento: 22/03/1985 \\
  Local Nascimento: Osasco - SP \\
  Nacionalidade: Brasileiro \\
  Estado Civil: Solteiro \\
  Fone: (19) 3229-4166 / 8849-9227 E-mail: isaacmechi@gmail.com \\
\end{flushleft}


\section{Educação}
\begin{CV}
\item[2/2008--12/2011 (Término previsto)] Faculdade Anhanguera de Campinas - Unidade IV
\\Bacharelado em Ciência da Computação
\end{CV}

\section{Cursos}
\begin{itemize}

	\item Cursando Inglês Pre Intermediate I - Centro Cultural Brasil Estados Unidos Campinas - Unidade Cambuí.
	\item Soluções Coorporativa em TI Sistemas Distribuídos e Arquitetura Mainframe: Básico (120 horas) – Instituto Eldorado.
	\item Soluções Coorporativa em TI Conceitos e Fundamentos (30 horas) – Instituto Eldorado.
	\item Curso Linux Básico e Avançado (40 Horas) – Instituto de Computação.
	\item Curso De Redes (20 Horas) – Instituto de Computação.	
	\item Curso de Fundamentos de Rede (20 Horas) – Fundação Bradesco.
	\item Curso de Html (37 Horas) – Fundação Bradesco.
	\item Curso de Montagem e Manutenção de Microcomputador – Senai Amoreiras.
%	\item Curso de Informática – CTAP Sesi Amoreiras.	

\end{itemize}

%\section{Eventos e Congressos}
%\begin{itemize}
%	\item Eventos que assistiu
%\end{itemize}

\section{Experiência Profissional}

\begin{CV}

\item[Desde 08/2011] \textbf{Função:} Analista de Sistemas Jr.
\\ \textbf{Empresa :} Motiva Contact Center
\\ \textbf{Atividades :} Gerência de usuários em MS-Windows Server 2003/2008, administração de redes e alguns  
servidores em ambiente GNU/Linux.

\item[07/2010--08/2011] \textbf{Função:} Estagiário Suporte Técnico
\\ \textbf{Empresa :} Instituto de Computação - Unicamp
\\ \textbf{Atividades :} Suporte técnico a usuários em ambientes MS-Windows e GNU/Linux, administração de redes e alguns  
servidores em ambiente GNU/Linux.
%\begin{itemize}
%	\item \textbf{Preto} sempre;
%	\item \textbf{Colorido} quase nunca.
%\end{itemize}

\item[01/2010--06/2010] \textbf{Função:} Estagiário Informática
\\ \textbf{Empresa :} Archivum Comercial LTDA
\\ \textbf{Atividades :} Digitalização de documentos. Escolhas de configurações e formatos de mídia digital para documentos a serem digitalizados.

\item[05/2007--07/2009] \textbf{Função:} Teleoperador
\\ \textbf{Empresa :} Atento do Brasil
\\ \textbf{Atividades :} Atendimento aos clientes Telefônica e suporte Speedy.

\end{CV}

\section{Áreas de Conhecimento}
Algumas áreas em que possuo conhecimentos.
\begin{itemize}
	\item \textbf{Sistemas Operacionais} GNU/Linux (Redhat, Debian), MS-Windows (XP, Windows 7); Servidor http Apache; Acompanhamento de logs.
	\item \textbf{Linguagens de Programação} C, Java, Python, PHP, Shel script, SQL.
\end{itemize}
\vspace{2\baselineskip}
\noindent Campinas, \today
\\
\\Produzido com \small \LaTeXe

\end{document}

%Tabellen
\begin{table}[htbp] \centering%
\begin{tabular}{lll}\hline\hline
1 & 2 & 3 \\ \hline
1 & \multicolumn{2}{c}{2} \\
\hline
\end{tabular}
\caption{Titel\label{Tabelle: Label}}
\end{table}
